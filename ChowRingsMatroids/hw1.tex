\documentclass[12pt]{article}
\usepackage[margin=1in]{geometry}% Change the margins here if you wish.
\setlength{\parindent}{0pt} % This is the set the indent length for new paragraphs, change if you want.
\setlength{\parskip}{5pt} % This sets the distance between paragraphs, which will be used anytime you have a blank line in your LaTeX code.
\pagenumbering{gobble}% This means the page will not be numbered. You can comment it out if you like page numbers.

%------------------------------------

% These packages allow the most of the common "mathly things"
\usepackage{amsmath,amsthm,amssymb}

% This package allows you to add images.
\usepackage{graphicx}
\usepackage{float}

% These are theorem environments.  This should cover everything you need, and you should be able to tell what environment goes with what type of result, but please let me know if I've missed anything.
\newtheorem{proposition}{Proposition}[section]
\newtheorem{classtheorem}{Theorem}
\newtheorem{theorem}{Theorem}[section]
\newtheorem{challenge}[theorem]{Challenge}
\newtheorem{question}[theorem]{Question}
\newtheorem{problem}[theorem]{Problem}


\newtheorem{theorempiece}{Theorem}[theorem]
\newtheorem{classtheorempiece}{Theorem}[classtheorem]
\newtheorem{challengepiece}{Challenge}[theorem]
\newtheorem{questionpiece}{Question}[theorem]
\newtheorem{problempiece}{Problem}[theorem]

%These help to format the names of the results the way we are in class and in notes.
%\renewcommand*{\proposition}{\Roman{section}.\arabic{theorem}}
\renewcommand*{\thetheorem}{\arabic{section}.\arabic{theorem}}
\renewcommand*{\theclasstheorem}{\Alph{theorem}}
\renewcommand*{\thechallenge}{\arabic{section}.\arabic{theorem}}
\renewcommand*{\thequestion}{\arabic{section}.\arabic{theorem}}
\renewcommand*{\theproblem}{\arabic{section}.\arabic{theorem}}
\renewcommand*{\thetheorempiece}{\arabic{section}.\arabic{theorem}.\alph{theorempiece}}
\renewcommand*{\thechallengepiece}{\arabic{section}.\arabic{theorem}.\alph{challengepiece}}
\renewcommand*{\thequestionpiece}{\arabic{section}.\arabic{theorem}.\alph{questionpiece}}
\renewcommand*{\theproblempiece}{\arabic{section}.\arabic{theorem}.\alph{problempiece}}


% Should you need any additional packages, you can load them here. If you've looked up something (like on DeTeXify), it should specify if you need a special package.  Just copy and paste what is below, and put the package name in the { }.  
\usepackage{wasysym} %this lets me make smiley faces :-)

% Put the name of your paper here. It does not need to be a fancy name, but should tell the reader what is contained in the paper.
\title{Group Actions on Matroids}

% You are the author, put your name here.
\author{Nutan Nepal}

% You can change the date to be something other than the current date if you want.
\date{\today}

\begin{document}

\maketitle

Let $f$ be an action of $S_n$ on a$M$ by the linear extension of
its action on $\{x_1,\ldots,x_n\}$
% These set the counters for the section (of our notes; so section 1 is rhombuses, section 2 is kites, etc.) and theorem (or problem, challenge, etc.)  If you are only presenting a piece of a theorem that is not the first piece, you'll need to uncomment the last counter and adjust it appropriately.
\setcounter{section}{1}
\setcounter{theorem}{1}
%\setcounter{theorempiece}{1}


% Use the appropriate environment to get the name that you want.  
\begin{proposition}
For the graded ring $$A=R[x_1,\ldotp,x_n] = R \oplus R(x_1,\ldots,x_n)
\oplus R(x_ix_j\ |\ 0<i<j\leq n) \oplus \ldots$$
if $S_n$ acts on $A$ by the linear extension of its action on $\{x_1,\ldots,
x_n\}$.
\end{proposition}
% Every environment should have a \begin{} and \end{} where the inside of {} matches.
\begin{proof} % This environment prints "Proof." at the beginning and the box at the end.

\setcounter{figure}{14}
%
%\begin{figure}[H]% This environment labels and generates a figure separate from the text.  I used the [H] to place the figure exactly where I wanted it.  If this isn't important to the reading and understanding of your proof, you can let LaTeX choose where to put it by leaving out the [] altogether.  You can use other tags to control the placement of a figure, [t] for the top of a page and [b] for the bottom are convenient sometimes.
%\centering
%\includegraphics[width = .4\textwidth]{I_16_picture}
% This is how we include pictures in a LaTeX document.  In Overleaf, you need to add this file to the project.  Click on "Project" with the squares in the menu at the top. Choose "Files..." and probably upload from your computer. (If you have installed a LaTeX editor on your own machine, then make sure the file you want to include is in the same folder as your .tex document... or look up how to call other folders.)
% Once the file is part of the project, type the name of the file inside the {}.  You typically do not need the file type .pdf, .jpg, etc. but nothing bad happens when you include it.
% The [] can be left empty or used to enter a variety of different instructions, most commonly adjustments to the size of the image are here. You can use inches or centimeters or points, LaTeX knows many measurement systems. You can also specify relative lengths.  \textwidth is the width of the text in this document, and putting .65 in front shrinks it to 85% of the text width.  You can also give a scale rather than a width or height: [scale = .5] will shrink the entire image to half the size of the original.
%\label{fig:1.15}
% This is a label we can use to reference this figure... more helpful if you have lots of figures in your proof.
%\caption{$\overline{AE} \cong \overline{EC}$, $\overline{BE} \cong \overline{EF}$, $\overline{BH} \cong \overline{HC}$, and $\overline{AH} \cong \overline{HI}$.}
% I like to caption the pictures in my geometry proofs with the congruences that are "constructed in" to the figure.  I basically include the "givens" and allow the reader to verify the conclusions by reading the proof and looking at the diagram together.  You can of course make different choices.
%\end{figure}
%
\end{proof}




\end{document}

