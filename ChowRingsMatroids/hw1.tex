\documentclass[12pt]{article}
\usepackage[margin=1in]{geometry}% Change the margins here if you wish.
\setlength{\parindent}{0pt} % This is the set the indent length for new paragraphs, change if you want.
\setlength{\parskip}{5pt} % This sets the distance between paragraphs, which will be used anytime you have a blank line in your LaTeX code.
\pagenumbering{gobble}% This means the page will not be numbered. You can comment it out if you like page numbers.

%------------------------------------

% These packages allow the most of the common "mathly things"
\usepackage{amsmath,amsthm,amssymb}

% This package allows you to add images.
\usepackage{graphicx, tikz-cd, adjustbox}
\usepackage{float}

% These are theorem environments.  This should cover everything you need, and you should be able to tell what environment goes with what type of result, but please let me know if I've missed anything.
\newtheorem{proposition}{Proposition}[section]
\newtheorem{classtheorem}{Theorem}
\newtheorem{theorem}{Theorem}[section]
\newtheorem{challenge}[theorem]{Challenge}
\newtheorem{question}[theorem]{Question}
\newtheorem{problem}[theorem]{Problem}

%These help to format the names of the results the way we are in class and in notes.
%\renewcommand*{\proposition}{\Roman{section}.\arabic{theorem}}
\renewcommand*{\thetheorem}{\arabic{section}.\arabic{theorem}}
\renewcommand*{\theclasstheorem}{\Alph{theorem}}
\renewcommand*{\thechallenge}{\arabic{section}.\arabic{theorem}}
\renewcommand*{\thequestion}{\arabic{section}.\arabic{theorem}}
\renewcommand*{\theproblem}{\arabic{section}.\arabic{theorem}}
% Put the name of your paper here. It does not need to be a fancy name, but should tell the reader what is contained in the paper.
\title{Group Actions on Matroids}

% You are the author, put your name here.
\author{Nutan Nepal}

% You can change the date to be something other than the current date if you want.
\date{\today}

\begin{document}

\maketitle

\hspace*{5mm} Any matroid $M$ can be thought of as a geometric
lattice (i.e. atomic and semimodular). An action of $S_n$ on a matroid $M$ over $n$-elements ground set $E=\{x_1,\ldots,x_n\}$ is defined
as the natural extension of its action on $E$
such that flats are mapped to flats. We note that such an action has the following properties
on the lattice:

\begin{enumerate}
    \item $g\cdot(a \vee b) = g\cdot a\vee g\cdot b.$
    \item $g\cdot(a \wedge b) = g\cdot a\wedge g\cdot b.$
    \item rk $(g\cdot a)$ = rk $a$.
    \item $A\subseteq B \iff g\cdot A\subseteq g\cdot B$.
\end{enumerate}

Let $\mathfrak{F}=\{F_1,\ldots,F_k\}$ be the collection of non empty proper flats and $E=\{1,\ldots,n\}$ be the set of
atoms of the matroid $M=(E,\mathfrak{F})$. The action $S_n \times E\to E$ of $S_n$ on $E$ extends to an action $S_n \times M
    \to M$.


For the graded ring $$R = k \oplus k(x_{F_1},\ldots,x_{F_k})
    \oplus k(x_{F_i}x_{F_j}\ |\ 0<i\leq j\leq k)\oplus\cdots,$$

let $I$ be the ideal generated by the elements $x_{F_i}\cdot x_{F_j}$ for incomparable flats $F_i$ and $F_j$.
Let $J$ be the ideal of $R$ generated by the elements $\alpha_{i}=\sum_{F_j\ni i}x_{F_j}$.
Then the Chow ring $A(M)$ of $M$ is given as the quotient $R/(I+J)$.

% These set the counters for the section (of our notes; so section 1 is rhombuses, section 2 is kites, etc.) and theorem (or problem, challenge, etc.)  If you are only presenting a piece of a theorem that is not the first piece, you'll need to uncomment the last counter and adjust it appropriately.
\setcounter{section}{1}
\setcounter{theorem}{1}

\begin{proposition}
    The action of $S_n$ on the matroid $M$ over $E=\{x_1,\ldots,x_n\}$ induces an action on $R$ which stabilizes the ideals $I$ and $J$.
\end{proposition}
\begin{proof}
    Given an action $k \mapsto \sigma\cdot k$ for $k\in E$, we define the induced action of on $R$ as the linear extension of
    $x_F \mapsto x_{\sigma\cdot F}$ where $\sigma\cdot F = \sigma\cdot\{k_1,\ldots, k_r\}=\{\sigma\cdot k_1,\ldots,\sigma\cdot k_r\}$.
    We note that the action as the automorphism of the matroid induces an isomorphism of $k-$modules.

    \vspace*{3mm}
    We now show that the action stabilizes the ideals $I$ and $J$:
    $F_j$ contains $F_i$ if and only if $\sigma\cdot F_j$ contains $\sigma \cdot F_i$. Hence
    $$\sigma\cdot \alpha_k=\sum_{\sigma\cdot F_j \ni \sigma\cdot k}{\sigma\cdot F_j} =\alpha_{\sigma\cdot k}\in J.$$
    Similarly, since $F_i$ and $F_j$ are incomparable precisely when $F_i\nsubseteq F_j$ and $F_j\nsubseteq F_i$,
    we have $F_i$ and $F_j$ incomparable if and only if $\sigma\cdot F_i$ and $\sigma\cdot F_j$ incomparable.
    Thus, $x_{F_i}\cdot x_{F_j}\in I$ if and only if $x_{\sigma\cdot F_i}\cdot x_{\sigma\cdot F_j}\in I$.
\end{proof}

The ideal $I+J$ has a monic Grobner basis $\{g_1,\ldots,g_t\}$ with respect to a monomial ordering. Thus, by [CLO15-2.5-5.3],
$R/(I+J)$ is a free $\mathbb{Z}-$module and has a basis given by standard monomials.
The Chow ring has the monomial $\mathbb{Z}-$basis
$$FY:=\{x_{F_1}^{m_1}x_{F_2}^{m_2}\cdots x_{F_k}^{m_k}: (\emptyset=:F_0)\subsetneq F_1\subsetneq F_2\cdots\subsetneq F_k, \text{ and }
    m_i\leq rk(F_i)-rk(F_{i-1})-1\}$$
given by the $FY-$monomials.

\vspace*{3mm}
The total degree of the monomials
$$\sum_{i=1}^{k}{m_i}\leq\sum_{i=1}^{k}{(rk(F_i)-rk(F_{i-1})-1)}=r$$
when the rank of the matroid is $r+1$.
Thus $A(M)=\oplus_{i=1}^{r}{A^k(M)}$.

\vspace*{3mm}
The group $S_n$ acts on each graded piece of degree $k$ as
$$x_{F_1}^{m_1}\cdots x_{F_k}^{m_k}\mapsto
x_{\sigma\cdot F_1}^{m_1}\cdots x_{\sigma\cdot F_k}^{m_k}$$
since
the incomparability relations are preserved under group actions.

\vspace*{3mm}
\textbf{Braid$-3$:}

The non-empty proper flats have the following lattice structure.
The $FY-$monomials are given as:
\begin{enumerate}
    \item degree 0: $1$,
    \item degree 1: $x_E$; $x_{F}$ for all rank 2 flats $F$
    \item degree 2: $x_E^2$. 
\end{enumerate}

\adjustbox{scale=0.65,center}{
    \begin{tikzcd}
        124 && 135 && 25 && 16 && 34 && 236 && 456 && {\text{rank } 2} \\
        \\
        \\
        & 1 && 2 && 3 && 4 && 5 && 6 &&& {\text{rank } 1}
        \arrow[from=1-1, to=4-2]
        \arrow[from=1-1, to=4-4]
        \arrow[from=1-3, to=4-2]
        \arrow[from=1-3, to=4-6]
        \arrow[from=1-1, to=4-8]
        \arrow[from=1-3, to=4-10]
        \arrow[from=1-7, to=4-2]
        \arrow[from=1-7, to=4-12]
        \arrow[from=1-9, to=4-6]
        \arrow[from=1-9, to=4-8]
        \arrow[from=1-11, to=4-4]
        \arrow[from=1-11, to=4-6]
        \arrow[from=1-11, to=4-12]
        \arrow[from=1-13, to=4-8]
        \arrow[from=1-13, to=4-10]
        \arrow[from=1-13, to=4-12]
        \arrow[from=1-5, to=4-4]
        \arrow[from=1-5, to=4-10]
    \end{tikzcd}
}

In particular, for any rank 3 matroid, the analogous result
about the $FY-$monomials holds.

\vspace*{3mm}
Not all elements $\sigma\in S_6$ give an action $x_F\mapsto
x_{\sigma\cdot F}$. For example, $(1\ 2)$ doesn't give an action
on $M$ since $(1\ 6)\mapsto (2\ 6)$ and $(2\ 6)$ is not a flat
of $M$; so it doesn't give an action on the Chow ring either.
The braid matroid can be realized as the following graph.

\adjustbox{scale=0.7,center}{
    \begin{tikzcd}
        \bullet \\
        \\
        &&&& \bullet \\
        \bullet \\
        && \bullet
        \arrow["1"', no head, from=1-1, to=4-1]
        \arrow["2"', from=1-1, to=5-3]
        \arrow["4"', no head, from=4-1, to=5-3]
        \arrow["3", no head, from=1-1, to=3-5]
        \arrow["5"', no head, from=4-1, to=3-5]
        \arrow["6"', no head, from=5-3, to=3-5]
    \end{tikzcd}
}

The subgroup $S_4$ in $S_6$ corresponding to the rigid symmetry
of the tetrahedron acts on the matroid $B-3$. For example,
$(1\ 2\ 4)(3\ 6\ 5)$ is a rotation of the tetrahedron keeping
the right vertex fixed. The corresponding action on the
$FY^1$ monomials is given below:
\begin{enumerate}
    \item $x_{124}\mapsto x_{124},\ x_{135}\mapsto x_{236},\
    x_{236}\mapsto x_{456},\ x_{456}\mapsto x_{135};$
    \item $x_{25}\mapsto x_{34},\ x_{34}\mapsto x_{16},\
    x_{16}\mapsto x_{25};$
    \item $x_E\mapsto x_E$.
\end{enumerate}
Similarly, the reflection $(1\ 3)(4\ 6)$ also gives an action.

\vspace*{3mm}
%The permutation representation of $S_6$ here is the homomorphism
%$$\rho: S_6\to Aut(B_3)$$
\adjustbox{center}{
    \begin{tikzcd}
        && \bullet && \bullet \\
        \\
        \bullet &&&& \bullet
        \arrow["2"', no head, from=1-3, to=3-1]
        \arrow["3"', no head, from=3-1, to=3-5]
        \arrow["4"', no head, from=1-3, to=3-5]
        \arrow["1"', no head, from=1-3, to=1-5]
    \end{tikzcd}
}
Another lattice of flats of a different matroid arising from
the graph above
is given below. The degree 1 $FY-$monomials are $x_{12}$,
$x_{13}$, $x_{14}$, $x_{234}$ and $x_E$.\\

\adjustbox{scale=0.7, center}{
    \begin{tikzcd}
        12 && 13 && 14 && 234 && {\text{rank } 2} \\
        \\
        1 && 2 && 3 && 4 && {\text{rank } 1}\\
        \arrow[from=1-1, to=3-1]
        \arrow[from=1-3, to=3-1]
        \arrow[from=1-1, to=3-3]
        \arrow[from=1-3, to=3-5]
        \arrow[from=1-7, to=3-7]
        \arrow[from=1-5, to=3-1]
        \arrow[from=1-7, to=3-3]
        \arrow[from=1-7, to=3-5]
        \arrow[from=1-5, to=3-7]
    \end{tikzcd}
}

The symmetries of the triangle are the automorphisms of the
matroid.
\end{document}