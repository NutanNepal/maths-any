\documentclass[12pt]{article}
\usepackage[margin=1in]{geometry}% Change the margins here if you wish.
\setlength{\parindent}{0pt} % This is the set the indent length for new paragraphs, change if you want.
\setlength{\parskip}{5pt} % This sets the distance between paragraphs, which will be used anytime you have a blank line in your LaTeX code.
\pagenumbering{gobble}% This means the page will not be numbered. You can comment it out if you like page numbers.

%------------------------------------

% These packages allow the most of the common "mathly things"
\usepackage{amsmath,amsthm,amssymb}
\usepackage{sectsty}

% This package allows you to add images.
\usepackage{graphicx, tikz-cd, adjustbox}
\usepackage{float}
\usepackage{scalerel}
\usepackage{stackengine,wasysym}

\newcommand\reallywidetilde[1]{\ThisStyle{%
  \setbox0=\hbox{$\SavedStyle#1$}%
  \stackengine{-.1\LMpt}{$\SavedStyle#1$}{%
    \stretchto{\scaleto{\SavedStyle\mkern.2mu\AC}{.5150\wd0}}{.6\ht0}%
  }{O}{c}{F}{T}{S}%
}}

% These are theorem environments.  This should cover everything you need, and you should be able to tell what environment goes with what type of result, but please let me know if I've missed anything.
\newtheoremstyle{mytheoremstyle} % name
    {\topsep}                    % Space above
    {\topsep}                    % Space below
    {}                   % Body font
    {}                           % Indent amount
    {\bf}                   % Theorem head font
    {.}                          % Punctuation after theorem head
    {.5em}                       % Space after theorem head
    {}  % Theorem head spec (can be left empty, meaning ‘normal’)

\theoremstyle{mytheoremstyle}

\newtheorem{proposition}{Proposition}[section]
\newtheorem{example}{Example}
\newtheorem{classtheorem}{Theorem}
\newtheorem{theorem}{Theorem}[section]
\newtheorem{challenge}[theorem]{Challenge}
\newtheorem{question}[theorem]{Question}
\newtheorem{problem}[theorem]{Problem}
\sectionfont{\fontsize{14}{15}\selectfont}
\subsectionfont{\fontsize{12}{15}\selectfont}

\newcommand{\bP}{\mathbb{P}}
\newcommand{\bA}{\mathbb{A}}
\newcommand{\bZ}{\mathbb{Z}}
\newcommand{\bC}{\mathbb{C}}
\newcommand{\bQ}{\mathbb{Q}}
\newcommand{\cO}{\mathcal{O}}
\newcommand{\cF}{\mathcal{F}}
\newcommand{\cM}{\mathcal{M}}
\newcommand{\ds}{\displaystyle}
\newcommand{\al}{\alpha}
\newcommand{\li}{l^{\infty}}
\newcommand{\ep}{\varepsilon}
\newcommand{\de}{\delta}
\newcommand{\spec}{\text{Spec\hspace*{.5mm}}}
\newcommand{\mspec}{\text{mSpec\hspace*{.5mm}}}
\newcommand{\linf}{l^{\infty}}
\newcommand{\cS}{\mathcal{S}}
\newcommand{\cR}{\mathcal{R}}
\newcommand{\cN}{\mathcal{N}}
 
%These help to format the names of the results the way we are in class and in notes.
%\renewcommand*{\proposition}{\Roman{section}.\arabic{theorem}}
\renewcommand*{\thetheorem}{\arabic{section}.\arabic{theorem}}
\renewcommand*{\theclasstheorem}{\Alph{theorem}}
\renewcommand*{\thechallenge}{\arabic{section}.\arabic{theorem}}
\renewcommand*{\thequestion}{\arabic{section}.\arabic{theorem}}
\renewcommand*{\theproblem}{\arabic{section}.\arabic{theorem}}
% Put the name of your paper here. It does not need to be a fancy name, but should tell the reader what is contained in the paper.
\title{Flat Modules}

% You are the author, put your name here.
\author{Nutan Nepal}

% You can change the date to be something other than the current date if you want.
\date{\today}

\begin{document}
\maketitle

\makebox[\linewidth]{\rule{200mm}{1pt}}
\section{Introduction}
\hspace*{8mm}In the study algebraic sets, we can consider a family of varieties as
follows: using the coordinate ring $R$ of an
affine variety $Y$ over an algebraically closed field $k$ and a collection
of polynomials $f_i(x_1,\ldots,x_n; b)\in R[x_1,\ldots,x_n]$, we consider
the collection of zero sets
$$V(f_i(x;b))=\{a\in \mathbb{A}^n_k|\ f_i(a,b)=0\},\ \ \ b\in Y$$ to be
in \textbf{a family parametrized by $Y$}. The polynomials $f_i$ were not over the
field k and the zero sets need not be algebraic sets themselves. However,
they have covering by affine varieties and can be considered
as subschemes of $\bA^n_k$.
If $I$ is the ideal generated by the polynomials $f_i$'s
in the ring $R[x_1,\ldots,x_n]$,
then we get the ring morphism $f:R\to R[x_1,\ldots,x_n]/I$.
This ring morphism induces the morphism of the maximal spectrums
$$\mspec (R[x_1,\ldots,x_n]/I) \to \mspec R = Y$$
which describes the maps between the subschemes $V(I)$ and $Y$.
The fiber over each point $b\in Y$ is then given by
$$\mspec (R[x_1,\ldots,x_n]/I\otimes R/m_b)=\mspec (k[x_1,\ldots,x_n]
/(f_i(x,b))).$$
The well-behavedness of the fibers over Y depends on the family of
the subschemes i.e. on the morphisms of the subschemes.
\begin{example}
    The projection
    $$\{(x,y,t)\in \bA^3_k\mid y-xt=0\}\to \bA^2_k$$
    defines a family of subvarieties of the $V(y-xt)$.
    The morphism has fiber over $(0,0)$ as one-dimensional variety
    (since any value of $t$
    satisfies the polynomial) and the fibers over other points are
    zero-dimensional.$\blacktriangle$
\end{example}

Hence, a morphism of varieties $f:X\to Y$ itself defines a family of subvarieties
of the source parametrized by the target. The families where the
fibers vary ``nicely'' are called \textbf{flat families} and these
families are characterized by \textbf{flat morphisms} between
the subschemes. The notion of flatness was first introduced by
Serre for algebraic reasons and Grothendieck later recognized the
geometric significance of it.

\vspace*{2mm}
\hspace*{8mm}
Similar considerations with homogenous polynomials in $R[x_0,\ldots,x_n]$
defines a family of subschemes of $\bP^n_k$.
Here we will try to describe the notion of flat morphisms
between two schemes.

\section{Flat modules and flat maps}

\hspace*{8mm}For any $A-$module $M$, every short exact sequence
$0\longrightarrow N'\longrightarrow N\longrightarrow N''
\longrightarrow 0$ of $A-$ modules induces the exact sequence
$ N'\otimes M\longrightarrow N\otimes M
\longrightarrow N''\otimes M\longrightarrow 0$.
An $A-$module $M$ is flat if for every short exact sequence
$\mathcal{S}:$
$0\longrightarrow N'\longrightarrow N\longrightarrow N''
\longrightarrow 0$, the induced sequence $\mathcal{S}\otimes M:$ 
$$0\longrightarrow N'\otimes M\longrightarrow N\otimes M
\longrightarrow N''\otimes M\longrightarrow 0$$
is also exact. $M$ is called faithfully flat if we have:
$\mathcal{S}\text{ exact}\iff \mathcal{S}\otimes M \text{ exact}.$
A ring homomorphism $f:R \to S$ is called flat (resp. faithfully flat)
if $S$ is flat (resp. faithfully flat) as
an $R-$module.

\subsection*{Examples and facts:}
\begin{enumerate}
    \item Projective modules are flat. Injective modules need not be flat
        and flat modules need not be projective or injective.
    \item $\mathbb{Z}$ is a projective (and hence, flat) $\bZ-$module.
        In general, $-\otimes_A A$ is the identity endofunctor
        the category of $A-$modules
        i.e. $\cS\otimes_A A = \cS$ and so,
        $A$ is always a flat $A-$module.
    \item The localization $S^{-1}N$ of an $A-$module $N$ by the multiplicative
        set $S$ of $A$ is an exact functor . So $A\to S^{-1}A$ is a flat ring
        morphism.
    \item $\bQ$ is a flat $\bZ-$module. ($\bQ$ is also an injective
        $\bZ-$module)
    \item Flat modules are torsion-free. Hence, the $\bZ-$modules
        $\bZ/n\bZ$ and $\bQ/\bZ$ (which is injective) are not flat.
    \item Finitely generated modules over principal ideal domains are
        flat if and only if they are torsion-free if and only if they
        are free.
    \item The direct sum of flat modules are flat (since the tensor
        product commutes with arbitrary direct sums). In particular,
        the $\bZ-$module $\bQ\oplus\bZ$ (which is neither projective
        nor injective) is flat.
    \item The localization $S^{-1}N$ of an $A-$module $N$ by the multiplicative
        set $S$ of $A$ is an exact functor. So $A\to S^{-1}A$ is a flat ring
        morphism.
    \item Flatness is preserved by change of base ring:
        If $M$ is a flat $B-$module and $B\to A$ is a ring morphism,
        then $M\otimes_B A$ is a flat $A-$module.
    \item Flatness is preserved by composition:
        If $A$ is a flat $B-$algebra and $M$ is a flat $A-$module, then
        $M$ is also $B-$flat.
    \item Flatness is a local property:
        An $A-$module $M$ is $A-$flat if and only if $M_\mathfrak{p}$
        is $A_\mathfrak{p}-$flat for all prime ideals $\mathfrak{p}
        \subset A$.
    \item An $A-$module $M$ is $A-$flat if and only if for all ideals $I$
        of $A$, we have $I\otimes M\cong IM$.
\end{enumerate}

\section{Quasicoherent sheaves and flat morphisms}
\hspace*{8mm}Given a ring $R$ and an 
$R-$module $M$, we can form a sheaf of abelian
groups $\widetilde{M}$ on $X= \spec R$ by taking $\widetilde{M}(D(f))
= M\otimes f^{-1}R = f^{-1}M$, which is the localization of $M$ by the
multiplicative set $\{1,f,f^2,\ldots\}$, and then extending to
all the open sets. The sheaf $\widetilde{M}$
has the structure of an $\cO_X-$module.

\begin{enumerate}

    \item A sheaf $\cF$ on $X$ is called \textbf{quasicoherent} if for each affine
        open subset Spec $A$ of $X$, the sheaf $\cF\mid_{\spec A}$
        is isomorphic to $\widetilde{M}$ for some $A-$module $M$.
        The category of quasicoherent sheaves over an affine
        scheme $\spec A$ is equivalent to the category of $A-$modules:
        $\mathcal{QC}oh_{\spec A} \tilde{\longleftrightarrow}
        \mathfrak{mod}_A$.
    
    \item A \textbf{quasicoherent sheaf} $\cF$ on $X$ is said to be
        \textbf{flat} at $p\in X$
        if $\cF_p$ is a flat $\cO_{X,p}-$module.
        A \textbf{quasicoherent sheaf} $\cF$ on $X$ is said to be
        \textbf{flat} over $X$
        if for every point $p\in X$, $\cF_p$ is a flat
        $\cO_{X,p}-$module.

    \item A morphism $f:X\to Y$ of schemes is said to be \textbf{flat}
        at $p\in X$ if $\cO_{X,p}$ is a flat
        $\cO_{Y,f(p)}-$module.
        A morphism $f:X\to Y$ of schemes is called a \textbf{flat
        morphism} if for every $p\in X$, $\cO_{X,p}$ is a flat
        $\cO_{Y,f(p)}-$module.
        A morphism is called \textbf{faithfully flat} if it is both
        flat and surjective.

        In particular, if $Y=\mspec A$, a morphism (or family)
        of schemes
        $f:X\to Y$ is flat if $\cO(U)$ is a flat $A-$module for every
        open set $U\subset X$.

\end{enumerate}

\subsection*{Examples and facts}
\begin{enumerate}
    \item Given a ring map $B\to A$, $A$ is faithfully flat over $B$
        if and only if $\spec A\to\spec B$ is a faithfully flat morphism.
    \item Open embeddings are flat. $f:X\to Y$ is an \textbf{open
        embedding} if $X$ is isomorphic to an open set of $Y$. For a
        morphism $f:D(y-x^2)\to \bA^2_k$ which is clearly an open
        embedding we see that 
    \item A morphism of rings $A\to B$ is flat if and only if the
        corresponding morphism of schemes $\spec B\to\spec A$.
        More generally, if $B\to A$ is a ring homomorphism and $M$ is
        an $A-$module, then $M$ is $B-$flat if and only if
        $\widetilde{M}$ is flat over $\spec B$.
    \item The fibers of a flat morphism of varieties $f:X\to Y$ all
        all have the same dimension dim $X-$ dim $Y$. 
    \item If $f:X\to Y$ is a surjective morphism of a variety to
        a non-singular curve, then it is flat.
    \item The above two statements imply: The fibers of a surjective
        morphism of a variety to a non-singular curve have dimension
        equal to dim $X-1$.
\end{enumerate}

\newpage
\section*{References}
\begin{enumerate}
    \item Arapura, Donu. Algebraic Geometry over the complex numbers. Springer 2012
    \item Vakil, Ravi. Foudations of Algebraic Geometry (2023 draft)
    \item The Stacks Project. https://stacks.math.columbia.edu/tag/00H9
    \item Matsumura, Hideyuki. Commutative Ring Theory. Cambridge University Press. 2006
\end{enumerate}

\setcounter{section}{1}
\setcounter{theorem}{1}

\end{document}